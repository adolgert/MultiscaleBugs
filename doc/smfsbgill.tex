\documentclass{article}
\title{Looking at the \texttt{gillespie} function in smfsb}
\date{\today}
\begin{document}
\maketitle

$x$ is the initial marking for the system. It will be a vector
of integers.
The variables \texttt{Pre} and \texttt{Post} are matrices for
the left-hand side and right-hand side stoichiometries.
Let's call them $\xi_{-}$ and $\xi_{+}$. This
means we're thinking of chemical kinetics, and these matrices
represent a number of molecules of each species. I don't know
whether they are $\mbox{species}\times\mbox{balance equation}$
or the transpose of that.

The change in species count is $S=\xi_{+}-\xi_{-}$.

The trajectory will be in a matrix $X$ whose row count
is $n+1$ time steps, where the $+1$ is for the initial marking.
The number of columns is the row count of the $\xi_{\pm}$,
so that must be the number of species. This tells us our first
guess was correct on orientation of $\xi$ and $S$.

Now for the loop which contains the Gillespie algorithm.
The function $h(x, t)$ returns a vector of $v$ rates
as a function of the current marking and current time.

The time is incremented by sampling an exponential with
a rate equal to the total rate of the system. Good.

The R \texttt{sample} function chooses among
the reactions, weighting them according
to the rates returned by $h(x,t)$. That's fine, too.
Each value in the marking is incremented according
to the stoichiometry of the chosen reaction.

So all of the bodies are buried in $h(x,t)$.

\end{document}
